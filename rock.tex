\documentclass{scrartcl}

\usepackage{amsmath}
%\usepackage{siunitx}

\begin{document}
  \title{R\"ocke schneidern 1.01}
  \author{Liana Stein}
\maketitle

\begin{enumerate}
  \item Aus einer Rolle Schneiderpapier ein ca.$\SI{70}{\centi\meter}$ langes
    und $\SI{60}{\centi\metmer}$ breites Stueck entnehmen.
  \item Das Papier so falten, dass man ein $\SI{70}{\centi\meter}\times
    \SI{30}{\centi\meter}$ grosses Rechteck erhaelt.
  \item Nun markiert man die rechte untere Seite mit $\mathcal{A}$.
  \item Nun um ein viertel der Taillenweite senkrecht nach oben die Stelle mit 
    $\mathcal{B}$ markieren. In diesem Beispiel hat die Strecke
    $\bar{\mathcal{AB}}$ also $\SI{72}{\centi\meter}:4 = 
    \SI{18}{\centi\meter}$.
  \item Auf der linken Seite markiert man den Punkt $\mathcal{C}$, sodass
    die Laenge $\mathcal{AC}$ die gewuenschte Laenge des Rocks ist. Hier ist
    die maximale Laenge natuerlich die Papierlange $70 cm$.
  \item Nun verbindet man mit einem Lineal die Punkte $\mathcal{A,B,C und D}$
  \item Die Linie $\mathcal{AE}$ misst $\frac{1}{4}$ der Taillenweite plus
    $2 cm$ fuer den \emph{Abnaeher} (keilfoermige Naht, mit denen ein
    Kleidungsstueck koerpernah geformt werden kann). In diesem Fall:
    \begin{align}
      72 cm / 4 &= 18 cm \\
      		&+ 2 cm \\
		&= 20 cm
      \label{abnaeher}
    \end{align}
  \item $\mathcal{BF}$ hat ??????????????????
  \item 
  \item
  \item
  \item
  \item
\end{enumerate}
\end{document}
