\documentclass{scrartcl}

\usepackage{amsmath}
\usepackage{siunitx}
\usepackage[utf8]{inputenc}
\usepackage{marvosym}
%\usepackage{bijan_commands}
\usepackage[ngerman]{babel}

% original by: Jeanette Ignacio da Silva

\begin{document}
  \title{R"ocke schneidern 1.01}
  \author{Jeanette \& Liana}
  \date{}
\maketitle

  
\noindent \mbox{} \hspace{1cm} \texttt{Taillenweite:} $ \SI{72}{\centi\meter} $ \\
\mbox{} \hspace{1cm} \texttt{H"uftweite:} ~~~~ $ \SI{108}{\centi\meter}$ \\
\mbox{} \hspace{1cm} \texttt{Rockl"ange:} ~~~~ $ \SI{58}{\centi\meter} $ 

\begin{enumerate}
  \item Aus einer Rolle Schneiderpapier ein ca. $\SI{70}{\centi\meter}$ langes
    und $\SI{60}{\centi\meter}$ breites St"uck entnehmen.

  \item Das Papier so falten, da"s man ein $\SI{70}{\centi\meter}\times
    \SI{30}{\centi\meter}$ gro"ses Rechteck erh"alt.
 
  \item Nun markiert man die rechte untere Seite mit $\mathcal{A}$.
 
  \item Nun um ein viertel der H"uftweite senkrecht nach oben die Stelle mit 
    $\mathcal{B}$ markieren. In diesem Beispiel hat die Strecke
    $\overline{\mathcal{AB}}$ also $\SI{108}{\centi\meter}/4 = 
    \SI{27}{\centi\meter}$.
 
  \item Auf der linken Seite markiert man den Punkt $\mathcal{C}$, soda"s
    die L"ange $\mathcal{AC}$ die gew"unschte L"ange des Rocks ist. Hier ist
    die maximale L"ange nat"urlich die Papierl"ange $\SI{70}{\centi\meter}$.
  
  \item Nun verbindet man mit einem Lineal die Punkte 
    $\mathcal{A,B,C} \text{ und } \mathcal{D}$.
  
  \item Die Linie $\overline{\mathcal{AE}}$ misst $\frac{1}{4}$ der Taillenweite
    plus $\SI{2}{\centi\meter}$ f"ur den \emph{Abn"aher} (keilf"ormige Naht,
    mit  denen ein Kleidungsst"uck k"orpernah geformt werden kann). In diesem
    Fall: \vspace{-2em}

      \begin{align*}
	\SI{72}{\centi\meter}/4=&~\SI{18}{\centi\meter} \\
	\SI{18}{\centi\meter}+\SI{2}{\centi\meter}=&~\SI{20}{\centi\meter}
	\label{abnaeher}
      \end{align*}

  \item Die Strecke $\overline{\mathcal{BF}}$ entspricht der H"ohe
    der Taille bis zum breiteren Teil der H"ufte.

  \item Mit einem Kurvenlineal die Punkte $\mathcal{E}\text{ und }
      \mathcal{F}$ verbinden, und dabei die Form der H"ufte festlegen.
 
  \item Nun setzt man einen neuen Punkt $\mathcal{G}$ einen Zentimeter links
      von $\mathcal{A}$. Mit einem Kurvenlineal die Punkte
      $\mathcal{G}\text{ und }\mathcal{E}$ verbinden, wobei dies die Form der
      vorderen Taille festlegt.

  \item Die Strecke $\overline{\mathcal{AH}}$, die sich "uber der Linie 
    $\overline{\mathcal{AB}}$ befindet, ist $\frac{1}{12}$ der
    Taillenweite. In unserem Fall $\SI{6}{\centi\meter}$. 

  \item Jetzt eine Gerade parallel zu $\overline{\mathcal{AC}}$ zeichnen die
      bei $\mathcal{H}$ startet und $\SI{8}{\centi\meter}$ lang ist, was der
      L"ange des Abn"ahers entspricht.% und sich der Punkt $\mathcal{I}$ befindet.

    \item Die Punkte $\mathcal{J}\text{ und }\mathcal{J'}$ befinden sich einen 
      Zentimeter rechts und links von $\mathcal{H}$ entfernt und entsprechen der 
      Tiefe des Abn"ahers wo sich der Punkt  $\mathcal{I}$ befindet.
     
    \item Mit einem Kurvenlineal die Punkte $\mathcal{J,I}$ und $\mathcal{J',I}$
      verbinden.

    Der vordere Teil der Form ist fertig und wir ziehen die Kontur mit
    einem dicken Kugelschreiber nach, indem wir den Linien
    $\overline{\mathcal{GEFDC}}$, $\overline{\mathcal{JI}}$ und 
    $\overline{\mathcal{J'I}}$ folgen. 
    F"ur den hinteren Teil nehmen wir einen feinen Stift mit anderer Farbe
    um die Zeichnungen nicht zu verwechseln.

  \item Der Punkt $\mathcal{L}$ befindet sich $\SI{2}{\centi\meter}$
    links von $\mathcal{A}$, "uber der Linie $\overline{\mathcal{AC}}$.

  \item Vereine nun die Punkte $\mathcal{L}$ mit $\mathcal{E}$ mithilfe des
    Kurvenlineals.  

  \item Ziehe nun eine Senkrechte Linie mit $\SI{10}{\centi\meter}$ von
    $\mathcal{H}$ um den Punkte $\mathcal{M}$ zu bekommen, welches die H"ohe des 
    hinteren Abn"ahers wird.

  \item Jetzt mit einem Kurvenlineal die Punkte $\mathcal{J-M}$ und
    $\mathcal{J'-M}$ verbinden.

  \item Ziehe das Kopierr"adchen "uber die Linien $\overline{\mathcal{JM}}$,
    $\overline{\mathcal{J'M}}$ und $\overline{\mathcal{EL}}$ um den Teil der
    R"uckseite auf dem Papier zu markieren.

  \item Schneide das noch gefaltete Papier entlang der Linie
    $\overline{\mathcal{GEFDC}}$. 

  \item Nun schneide das Papier am Falz um den vorderen vom hinteren Teil zu
    trennen. 

  \item Fahre nun mit einem dicken andersfarbigen Stift die Linien des hinteren
    Rockteiles $\overline{\mathcal{JM}}$, $\overline{\mathcal{J'M}}$ und
    $\overline{\mathcal{EL}}$ nach.

  \item Schneide die untere Vorlage entlang der Linie $\overline{\mathcal{EL}}$.

\end{enumerate}

\paragraph{Damit} ist die hintere Vorlage fertig und somit auch die Grundvorlage
f"ur den Rock. F"uer die Markierung legen wir die Vorlage auf die 
Stoffr"uckseite und achten dabei auf den Fadenlauf. Dieser muss in 
L"angsrichtung verlaufen damit der Rock sp"ater sch"on f"allt.
Das ganze wird mit N"ahnadeln befestigt und mit dem Kopierr"adchen auf
Kopierpapier umfahren. Dann nur noch mit $\SI{2}{\centi\meter}$ Abstand
ausschneiden und \emph{\dots fertig!}

\end{document}
