\documentclass{scrartcl}

\usepackage{amsmath}
\usepackage{siunitx}
\usepackage[utf8]{inputenc}
\usepackage{marvosym}
%\usepackage{bijan_commands}
\usepackage[ngerman]{babel}

% original by: Jeanette Ignacio da Silva

\begin{document}
%   \title{R"ocke schneidern 1.01}
%   \author{}
%   \date{}
% \maketitle
% \vspace{-2cm}
\centering{\Huge{\textbf{R"ocke schneidern 1.01}}}  
\\ \mbox{}\\
\flushleft
\noindent \mbox{} \hspace{1cm} \texttt{Taillenweite:} $ \SI{72}{\centi\meter} $ \\
\mbox{} \hspace{1cm} \texttt{H"uftweite:} ~~~~ $ \SI{108}{\centi\meter}$ \\
\mbox{} \hspace{1cm} \texttt{Rockl"ange:} ~~~~ $ \SI{58}{\centi\meter} $ 

\begin{enumerate}
  \item Aus einer Rolle Schneiderpapier ein ca. $\SI{70}{\centi\meter}$ breites
    und $\SI{60}{\centi\meter}$ langes St"uck entnehmen.

  \item Das Papier nach oben so falten, da"s man ein $\SI{70}{\centi\meter}\times
    \SI{30}{\centi\meter}$ gro"ses Rechteck erh"alt.
 
  \item Nun markiert man die rechte untere Seite mit $\mathcal{A}$.
 
  \item Nun um ein viertel der H"uftweite senkrecht nach oben die Stelle mit 
    $\mathcal{B}$ markieren. In diesem Beispiel hat die Strecke
    $\overline{\mathcal{AB}}$ also $\SI{108}{\centi\meter}/4 = 
    \SI{27}{\centi\meter}$.
 
  \item Auf der unteren Kante, links von $\mathcal{A}$ markiert man den Punkt
    $\mathcal{C}$, soda"s die L"ange $\mathcal{AC}$ die gew"unschte L"ange des 
    Rocks ist. Hier ist die maximale L"ange nat"urlich die Papierl"ange
    $\SI{70}{\centi\meter}$.
  
  \item Von $\mathcal{C}$ aus wieder $1/4$ der H"uftweite nach oben gehen und
    dort den Punkt $\mathcal{D}$ setzten, soda"s die vier Punkte ein Rechteck
    bilden.
  
  \item Nun verbindet man mit einem Lineal die Punkte 
    $\mathcal{A,B,C} \text{ und } \mathcal{D}$.
  
  \item Der Punkt $\mathcal{E}$ befindet sich an der rechten Kante senkrecht
    "uber $\mathcal{A}$, wobei die Linie $\overline{\mathcal{AE}}$ $1/4$ 
    der Taillenweite  plus $\SI{2}{\centi\meter}$ f"ur den \emph{Abn"aher}
    (keilf"ormige Naht,
    mit  denen ein Kleidungsst"uck k"orpernah geformt werden kann) misst. In 
    diesem Fall:% \vspace{-2em}

      \begin{align*}
	\SI{72}{\centi\meter}/4=&~\SI{18}{\centi\meter} \\
	\SI{18}{\centi\meter}+\SI{2}{\centi\meter}=&~\SI{20}{\centi\meter}
	\label{abnaeher}
      \end{align*}

  \item Der Punkt $\mathcal{F}$ liegt auf der Linie $\overline{\mathcal{BD}}$,   
	die Strecke $\overline{\mathcal{BF}}$ entspricht der L"ange zwischen
    	Taille und dem breitesten Teil der H"ufte.

  \item Mit einem Kurvenlineal die Punkte $\mathcal{E}\text{ und }
      \mathcal{F}$ verbinden, und dabei die Form der H"ufte festlegen.
 
  \item Nun setzt man einen neuen Punkt $\mathcal{G}$ einen Zentimeter links
      von $\mathcal{A}$. Mit einem Kurvenlineal die Punkte
      $\mathcal{G}\text{ und }\mathcal{E}$ verbinden, wobei dies die Form der
      vorderen Taille festlegt.

    \item Die Strecke $\overline{\mathcal{AH}}$, wobei $\mathcal{H}$ sich auf 
      der Linie $\overline{\mathcal{AB}}$ befindet, ist $\frac{1}{12}$ der
    Taillenweite. In unserem Fall also $\SI{6}{\centi\meter}$. 

  \item Jetzt eine Gerade parallel zu $\overline{\mathcal{AC}}$ zeichnen die
      bei $\mathcal{H}$ startet und $\SI{8}{\centi\meter}$ lang ist, und im 
      Punkt $\mathcal{I}$ endet. Dies entspricht der L"ange des Abn"ahers.

  \item Die Punkte $\mathcal{J}\text{ und }\mathcal{J'}$ befinden sich einen 
      Zentimeter rechts und links von $\mathcal{H}$ entfernt und entsprechen der 
      Tiefe des Abn"ahers. 
     
  \item Mit einem Kurvenlineal die Punkte $\mathcal{J,I}$ und $\mathcal{J',I}$
      verbinden.

\end{enumerate}


    \noindent Der vordere Teil der Form ist fertig und wir ziehen die Kontur mit
    einem dicken Kugelschreiber nach, indem wir den Linien
    $\overline{\mathcal{GEFDC}}$, $\overline{\mathcal{JI}}$ und 
    $\overline{\mathcal{J'I}}$ folgen. 
    F"ur den hinteren Teil nehmen wir einen feinen Stift mit anderer Farbe
    um die Zeichnungen nicht zu verwechseln.


\begin{enumerate}

  \item Der Punkt $\mathcal{L}$ befindet sich $\SI{2}{\centi\meter}$
    links von $\mathcal{A}$, auf der Linie $\overline{\mathcal{AC}}$.

  \item Vereine nun die Punkte $\mathcal{L}$ mit $\mathcal{E}$ mithilfe des
    Kurvenlineals.  

  \item Erweitere die Linie $\overline{\mathcal{HI}}$ um $\SI{2}{\centi\meter}$,
    um den Punkt $\mathcal{M}$ zu erhalten, welches die H"ohe des hinteren 
    Abn"ahers wird.

  \item Jetzt mit einem Kurvenlineal die Punkte $\mathcal{J-M}$ und
    $\mathcal{J'-M}$ verbinden.

  \item Ziehe das Kopierr"adchen "uber die Linien $\overline{\mathcal{JM}}$,
    $\overline{\mathcal{J'M}}$ und $\overline{\mathcal{EL}}$ um den Teil der
    R"uckseite auf dem Papier zu markieren.

  \item Schneide das noch gefaltete Papier entlang der Linie
    $\overline{\mathcal{GEFDC}}$. 

  \item Nun schneide das Papier am Falz um den vorderen vom hinteren Teil zu
    trennen. 

  \item Fahre nun mit einem dicken andersfarbigen Stift die Linien des hinteren
    Rockteiles $\overline{\mathcal{JM}}$, $\overline{\mathcal{J'M}}$ und
    $\overline{\mathcal{EL}}$ nach.

  \item Schneide die hintere Vorlage entlang der Linie $\overline{\mathcal{EL}}$.

\end{enumerate}

\noindent Damit ist auch die hintere Vorlage des Rockes fertig. F"ur die 
"Ubertragung der Vorlage auf den Stoff, m"u"sen wir diesen doppelt legen, mit
der "au"seren Seite auf links. Dabei mu"s der Fadenverlauf des Stoffes entlang 
der L"angsrichtung sein, damit der Rock sp"ater sch"on f"allt.
Das ganze wird mit N"ahnadeln befestigt und mit dem Kopierr"adchen auf
Kopierpapier umfahren. \\
Da die Vorderseite aus einem Teil besteht, legt man die Vorlage direkt an die 
Faltkante. Dann die anderen R"ander mit $\SI{2}{\centi\meter}$ Abstand
ausschneiden. Die R"uckseite besteht dagegen aus zwei Teilen und erh"alt auch
an der Faltkante einen $\SI{2}{\centi\meter}$ breiten Rand. \\ 
\mbox{} \\
\centering{\emph{Viel Spa"s beim Schneidern!}}
%\end{centering}

\end{document}
