\documentclass{scrartcl}

\usepackage{amsmath}
\usepackage{siunitx}
\usepackage[utf8]{inputenc}
\usepackage{marvosym}

% original by: Jeanette Ignacio da Silva

\begin{document}
  \title{R\"ocke schneidern 1.01}
  \author{Liana}
\maketitle

\begin{enumerate}
  \item Aus einer Rolle Schneiderpapier ein ca.$\SI{70}{\centi\meter}$ langes
    und $\SI{60}{\centi\meter}$ breites Stueck entnehmen.

  \item Das Papier so falten, dass man ein $\SI{70}{\centi\meter}\times
    \SI{30}{\centi\meter}$ grosses Rechteck erhaelt.
 
  \item Nun markiert man die rechte untere Seite mit $\mathcal{A}$.
 
  \item Nun um ein viertel der Taillenweite senkrecht nach oben die Stelle mit 
    $\mathcal{B}$ markieren. In diesem Beispiel hat die Strecke
    $\overline{\mathcal{AB}}$ also $\SI{72}{\centi\meter}/4 = 
    \SI{18}{\centi\meter}$.
 
  \item Auf der linken Seite markiert man den Punkt $\mathcal{C}$, sodass
    die Laenge $\mathcal{AC}$ die gewuenschte Laenge des Rocks ist. Hier ist
    die maximale Laenge natuerlich die Papierlaenge $\SI{70}{\centi\meter}$.
  
  \item Nun verbindet man mit einem Lineal die Punkte 
    $\mathcal{A,B,C} \text{ und } \mathcal{D}$.
  
  \item Die Linie $\overline{\mathcal{AE}}$ misst $\frac{1}{4}$ der Taillenweite
    plus $\SI{70}{\centi\meter}$ fuer den \emph{Abnaeher} (keilfoermige Naht,
    mit  denen ein Kleidungsstueck koerpernah geformt werden kann). In diesem
    Fall: \vspace{-1cm}

      \begin{align*}
	\SI{72}{\centi\meter}/4=&~\SI{18}{\centi\meter} \\
	\SI{18}{\centi\meter}+\SI{2}{\centi\meter}=&~\SI{20}{\centi\meter}
	\label{abnaeher}
      \end{align*}

  \item \Emailct ~\texttt{wo ist F? welche Breite?} %$\overline{\mathcal{BF}}$ 

  \item Mit einem Kurvenlineal die Punkte $\mathcal{E}\text{ und }
      \mathcal{F}$ verbinden, um die Form der Huefte festzulegen.
 
  \item Nun setzt man einen neuen Punkt $\mathcal{G}$ einen Zentimeter links
      von $\mathcal{A}$. Mit einem Kurvenlineal die Punkte
      $\mathcal{G}\text{ und }\mathcal{E}$ verbinden, wobei dies die Form der
      Taille festlegt.

    \item \Emailct \texttt{A-H = 6cm aber wo ist H?} \\ Die Strecke
      $\overline{\mathcal{AH}}$ ist $\frac{1}{12}$ der Taillenweite, also in
      unserem Fall $\SI{6}{\centi\meter}$. 
  \item
  \item
\end{enumerate}
\end{document}










